\subsection{Czujnik temperatury}
Czujnik MAX6576 zapewnia pomiar temperatury otoczenia rośliny.

\subsubsection{Konfiguracja mikrokontrolera}

Do obsługi czujnika MAX6576 wykorzystano pin \texttt{PIO1\_5} odpowiadający pinowi \texttt{P0.2}. Konfiguracja tego pinu przebiega w następujący sposób:

\begin{itemize}
    \item \texttt{PINSEL0} – bity 5:4 ustawione na 00, konfiguracja P0.2 jako funkcję GPIO.
    \item \texttt{FIODIR0} – bit 2 ustawiony na 0, co ustawia kierunek pinu P0.2 jako wejściowy.
    \item \texttt{FIOPIN0} – odczyt wartości logicznej pinu w celu wykrywania zboczy sygnału z czujnika.
\end{itemize}

\subsubsection{Inicjalizacja i odczyt}
\begin{enumerate}
    \item Konfiguracja pinu \texttt{PIO1\_5} (P0.2) jako wejściowego.
    \item System oczekuje na zbocze sygnału wejściowego, co inicjuje pomiar czasu.
    \item Zliczanie \texttt{NUM\_HALF\_PERIODS} (tutaj 340) kolejnych zboczy sygnału z czujnika.\\
    \item Pomiar czasu \(\Delta t_{\text{ms}}\) – liczba milisekund, które upłynęły między pierwszym a ostatnim zboczem.
    \item Przeliczenie czasu na temperaturę:
    \[
    10 \cdot T(\degree \mathrm{C}) = \frac{2 \cdot 1000 \cdot \Delta t_{\text{ms}}}{\texttt{NUM\_HALF\_PERIODS} \cdot \texttt{TEMP\_SCALAR\_DIV10}} - 2731
    \]
    \begin{itemize}
        \item \(\Delta t_{\text{ms}}\) – zmierzony czas (w ms) dla zadanej liczby półokresów.
        \item \texttt{TEMP\_SCALAR\_DIV10} – współczynnik zależny od konfiguracji pinów \texttt{TS0}/\texttt{TS1} (tutaj: 1).
        \item \texttt{NUM\_HALF\_PERIODS} – liczba półokresów poddanych pomiarowi (tutaj: 340).
        \item \(2731\) – stała przeliczająca wynik z kelwinów na dziesiąte części stopnia Celsjusza.
    \end{itemize}
    \item Wykorzystanie odczytanych danych – temperatura wyświetlana jest na ekranie OLED.
\end{enumerate}


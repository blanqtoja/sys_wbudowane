\documentclass{article}
\usepackage[T1]{fontenc}
\usepackage[utf8]{inputenc}
\usepackage[polish]{babel}
\usepackage{hyperref}

\title{Dokumentacja projektu: Inteligentna donica}
\author{
  Julia Dobroszek 251504 \\
  Aleksander Kaźmierczak 251544 (kierownik grupy)\\
  Malwina Wodnicka 251XXX \\  % Tutaj należy wstawić właściwy numer indeksu
}
\date{\today}

\begin{document}

\maketitle


\section{Instrukcja użytkowania}

\subsection{Uruchomienie urządzenia}
Aby uruchomić inteligentną donicę, należy wykonać następujące kroki:
\begin{enumerate}
    \item Podłączyć urządzenie do źródła zasilania za pomocą dołączonego przewodu.
    \item Umieścić czujnik wilgotności (dwa metalowe druty) w glebie w doniczce. Czujnik powinien być umieszczony na głębokości około 3-5 cm, aby zapewnić prawidłowy pomiar wilgotności gleby.
    \item Po podłączeniu zasilania, system automatycznie rozpocznie pracę i wyświetli informacje na ekranie OLED.
    
\end{enumerate}


\subsection{Interpretacja wskaźników}
Inteligentna donica dostarcza użytkownikowi informacji o stanie rośliny poprzez:
\begin{enumerate} 
     \item \textbf{Wyświetlacz OLED} - Na ekranie prezentowane są aktualne statystyki dotyczące: 
     \begin{itemize} 
         \item Poziomu wilgotności gleby:
             \begin{itemize}
                 \item \textbf{sucho} - oznacza niski poziom wilgotności gleby (odczyt powyżej 2801), wskazujący na konieczność podlania rośliny
                 \item \textbf{w normie} - oznacza optymalny poziom wilgotności gleby (odczyt między 1601 a 2800), idealny dla wzrostu większości roślin
                 \item \textbf{mokro} - oznacza wysoki poziom wilgotności gleby (odczyt poniżej 1600), wskazujący na nadmiar wody, który może być szkodliwy dla rośliny
             \end{itemize}
         \item Temperatury otoczenia (w stopniach Celsjusza) 
         \item Natężenia światła - system interpretuje odczyt z czujnika i wyświetla jeden z trzech komunikatów:
         \begin{itemize}
             \item "jasno" - gdy wartość odczytu przekracza 2000
             \item "OK" - gdy wartość odczytu mieści się w przedziale od 350 do 2000
             \item "ciemno" - gdy wartość odczytu jest mniejsza niż 350
         \end{itemize}
         \item Dodatkowych informacji o stanie systemu 
     \end{itemize} 
     
     \item \textbf{Dioda RGB} - W zależności od poziomu wilgotności gleby, dioda RGB świeci się różnymi kolorami: 
     \begin{itemize} 
         \item Kolor żółto-zielony (\#c8ff1e) - oznacza stan \textbf{sucho} (odczyt powyżej 2801), sygnalizując potrzebę podlania rośliny
         \item Kolor zielony (\#1eff00) - oznacza stan \textbf{w normie} (odczyt między 1601 a 2800), sygnalizując optymalny poziom wilgotności
         \item Kolor błękitny (\#1effff) - oznacza stan \textbf{mokro} (odczyt poniżej 1600), sygnalizując nadmiar wody w glebie
     \end{itemize} 
 \end{enumerate}

\subsection{Kalibracja i konserwacja}
\begin{enumerate}
    \item \textbf{Kalibracja} - System został skalibrowany fabrycznie na podstawie eksperymentów empirycznych. Urządzenie nie posiada możliwości zewnętrznej kalibracji przez użytkownika.
    
    \item \textbf{Konserwacja} - Aby zapewnić prawidłowe działanie urządzenia:
    \begin{itemize}
        \item Regularnie czyścić czujnik wilgotności z osadów mineralnych, które mogą gromadzić się na jego powierzchni
        \item Utrzymywać wyświetlacz OLED w czystości, unikając bezpośredniego kontaktu z wodą
        \item Sprawdzać okresowo stan połączeń elektrycznych
        \item \textbf{Kategorycznie unikać kontaktu elektroniki z wodą} - Jedynym elementem, który może mieć bezpośredni kontakt z wilgocią, jest czujnik wilgotności (metalowe druty). Pozostałe elementy elektroniczne należy chronić przed zalaniem, zachlapaniem oraz nadmierną wilgotnością powietrza, gdyż może to prowadzić do trwałego uszkodzenia urządzenia i utraty gwarancji
    \end{itemize}
\end{enumerate}

\section{Funkcjonalności}

\subsection{Magistrala SPI}
System wykorzystuje interfejs SPI do komunikacji z wyświetlaczem OLED oraz innymi peryferiami. Implementacja obejmuje konfigurację parametrów transmisji, takich jak częstotliwość zegara, tryb pracy oraz obsługę przerwań.

\subsection{Magistrala I2C}
\begin{enumerate}
    \item Interfejs I2C jest używany do komunikacji z czujnikiem światła oraz innymi komponentami wymagającymi komunikacji szeregowej. Implementacja obejmuje inicjalizację magistrali I2C, konfigurację adresów urządzeń oraz obsługę błędów komunikacji.
    \item Płytka LPC1769 posiada trzy magistrale I2C, w projekcie została użyta magistrala I2C2. 
    \item Czujnik światła jest podłączony do magistrali I2C2, a jego adres jest ustawiony na 0x44. Szczegóły konfiguracji znajdują się w sekcji \ref{sec:czujnik_swiatla}.
\end{enumerate}

\subsubsection{Ustawienia rejestrów}
Rejestry mikroprocesora zostały opisane w tabeli 384. I2C register map w dokumentacji płytki. Poniżej znajdują się opisane rejestry:
\begin{itemize}
    %sprawdzic czy dobry adres
    \item \textbf{I2CONSET} - Rejestr ustawień kontrolnych I2C. Używany do ustawiania bitów w rejestrze kontrolnym I2CON. Pozwala ustawiać pojedyncze bity bez konieczności odczytywania i modyfikowania całego rejestru. Zapisanie '1' do bitu w tym rejestrze ustawia odpowiadający bit w rejestrze kontrolnym I2C. Poniżej zostały opiasne flagi:
    \begin{enumerate}
        \item \textbf{-} - Bity 0 i 1 są zarezerowane 
        \item \textbf{AA} - Bit 2 \\
            Gdy:\\
            AA = 1 urządzenie wyśle ACK (czyli ustawi linię SDA w stan niski w czasie impulsu zegarowego na linii SCL). \\
            AA = 0 urządzenie wyśle NACK (czyli pozostawi linię SDA w stanie wysokim).
        \item \textbf{SI} - Bit 3 \\
            Flaga ta informuje o zmianie stanu magistrali np. zakończenie nadawania danych.
            Gdy:\\
            SI = 1, niski poziom sygnału SCL jest wydłużony. Transmisja zostaje wstrzymana.\\
            SI resetuje się poprzez zapisanie 1 do bitu SIC (SI Clear) w rejestrze I2CONCLR
        \item \textbf{STO} - Bit 4\\
            Flaga zakończenia transmisji danych. Jej działanie zależy od trybu, w jakim pracuje układ: master czy slave.\\
            W trybie master, ustawienie STO = 1 powoduje wysłanie warunku STOP na magistrali I2C, czyli SDA przechodzi z LOW na HIGH, gdy SCL jest HIGH. To kończy bieżącą transmisję i linia jest zwalniana.\\
            Gdy magistrala wykryje warunek STOP, flaga STO zostaje automatycznie wyczyszczona przez sprzęt.

        \item \textbf{STA} - Bit 5\\
            Służy do rozpoczęcia transmisji przez wysłanie warunku START
                \begin{itemize}
                    \item Jeżeli interfejs I2C jest w trybie slave, to:
                    \begin{itemize}
                        \item Interfejs przechodzi do trybu master
                        \item Jeśli magistrala jest wolna – natychmiast wysyła warunek START (SDA z HIGH do LOW przy SCL = HIGH).
                        \item Jeśli magistrala jest zajęta – czeka na warunek STOP, który „uwolni” magistralę, a potem wysyła START po pół okresie zegara wewnętrznego.
                    \end{itemize}

                    \item Jeśli interfejs już jest w trybie master, to:
                    \begin{itemize}
                        \item Ustawienie STA = 1 powoduje wysłanie repeated START – kolejnego warunku START bez wcześniejszego STOP.
                    \end{itemize}   
                    \item W przypadku, gdy STA oraz STO ustawimy jednocześnie na 1, to:
                    \begin{itemize}
                        \item w tybie master zostanie syłany sygnal STOP, następnie wysyłany jest START (restart transmisji)
                        \item w trybie slave nastąpi zatrzymanie transmisji bez wysłąnia STOP na magistralę, a następnie, jeśli magistrala jest wolna, zostanie wysłany START
                    \end{itemize}
                \end{itemize}
        \item \textbf{I2EN} - Bit 6\\
        Flaga odpowiedzialna za włączenie lub wyłączenie interfejsu I2C. Gdy:\\\
            I2EN = 0, interfejs I2C jest wyłączony\\
            I2EN = 1, interfejs I2C jest włączony
            
        
    \end{enumerate}
    \item \textbf{I2CONCLR} - Rejestr kasowania ustawień kontrolnych I2C. Używany do kasowania bitów w rejestrze kontrolnym I2C. Zapisanie '1' do bitu w tym rejestrze kasuje odpowiadający bit w rejestrze kontrolnym I2C.
    \begin{enumerate}
        \item AAC to bit w rejestrze I2CONCLR, który służy do wyzerowania bitu AA (Assert Acknowledge) w rejestrze I2CON. Bit AA odpowiada za to, czy urządzenie I2C nada ACK (potwierdzenie) podczas komunikacji.
    \end{enumerate}
    
    
    
    \item \textbf{I2STAT} - Rejestr statusu I2C (Adres: 0x08). Dostarcza szczegółowe kody statusu odzwierciedlające aktualny stan interfejsu I2C.
    \item \textbf{I2DAT} - Rejestr danych I2C (Adres: 0x0C). Podczas trybu nadawania lub odbioru, dane są zapisywane lub odczytywane z tego rejestru.
    \item \textbf{I2ADR0, I2ADR1, I2ADR2, I2ADR3} - Rejestry adresowe I2C (Adresy: 0x20, 0x24, 0x28, 0x2C). Zawierają 7-bitowy adres slave do działania interfejsu I2C w trybie slave.
    \item \textbf{I2SCLH} - Rejestr cyklu zegara SCL I2C (wysoka połowa) (Adres: 0x10). Zawiera wysoką połowę cyklu zegara I2C.
    \item \textbf{I2SCLL} - Rejestr cyklu zegara SCL I2C (niska połowa) (Adres: 0x14). Zawiera niską połowę cyklu zegara I2C.
    \item \textbf{MMCTRL} - Rejestr kontrolny trybu monitorowania I2C (Adres: 0x1C). Używany do kontrolowania trybu monitorowania I2C.
    \item \textbf{I2DATA\_BUFFER} - Rejestr bufora danych I2C (Adres: 0x2C). Zawiera zawartość 8 najstarszych bitów rejestru przesuwnego I2DAT.
    \item \textbf{I2MASK0, I2MASK1, I2MASK2, I2MASK3} - Rejestry maski I2C (Adresy: 0x30, 0x34, 0x38, 0x3C). Służą do określenia dopasowania adresu w trybie slave.
\end{itemize}

\subsubsection{Inicjalizacja magistrali I2C2}
Inicjalizacja magistrali I2C2 odbywa się poprzez wykonanie następujących kroków:
\begin{enumerate}
    %opis ustawien konkretnych rejestrow w projekcie, uzasadnienie czemu wybralismy takie ustawienia
    \item 


\end{enumerate}











\subsection{Czujnik światła}
\label{sec:czujnik_swiatla}
Inteligentna donica wykorzystuje czujnik światła oparty na protokole komunikacyjnym I2C. Implementacja tego elementu obejmuje trzy kluczowe funkcjonalności:

\begin{enumerate}
    \item \textbf{Inicjalizacja i konfiguracja czujnika ISL29003} - Proces inicjalizacji obejmuje ustawienie adresu urządzenia na magistrali I2C, konfigurację rejestrów czujnika oraz określenie parametrów pomiaru, takich jak czułość i częstotliwość próbkowania.

    \item \textbf{Odczyt danych z czujnika} - Funkcjonalność ta odpowiada za komunikację z czujnikiem poprzez protokół I2C, wysyłanie komend odczytu do odpowiednich rejestrów oraz interpretację otrzymanych danych. Implementacja uwzględnia obsługę błędów komunikacji oraz weryfikację poprawności odczytanych wartości.

    \item \textbf{Przetwarzanie i analiza danych} - Po odczytaniu surowych danych z czujnika, system przetwarza je na użyteczne informacje o poziomie natężenia światła. Funkcjonalność ta obejmuje kalibrację, filtrowanie zakłóceń oraz konwersję wartości na jednostki zrozumiałe dla użytkownika (np. luxy).
\end{enumerate}

\subsection{Czujnik temperatury}
System monitoruje temperaturę otoczenia rośliny za pomocą dedykowanego czujnika. Implementacja obejmuje:

\begin{enumerate}
    \item \textbf{Inicjalizacja czujnika temperatury} - Proces inicjalizacji obejmuje konfigurację czujnika z wykorzystaniem funkcji czasowych (getTicks) oraz ustawienie parametrów pomiaru.
    
    \item \textbf{Cykliczny odczyt temperatury} - Funkcjonalność ta odpowiada za regularne pobieranie danych z czujnika oraz konwersję odczytanych wartości na format zrozumiały dla użytkownika.
    
    \item \textbf{Wykorzystanie danych temperaturowych} - Informacja o temperaturze jest wyświetlana na ekranie OLED i może wpływać na decyzje systemu dotyczące nawadniania lub doświetlania rośliny.
\end{enumerate}

\subsection{Wyświetlacz OLED}
Donica wykorzystuje wyświetlacz OLED model UG-9664HSWAG01 (oznaczenie na schemacie OLED1) komunikujący się przez interfejs SPI. Wyświetlacz bazuje na sterowniku SSD1305 132 x 64 Dot Matrix OLED/PLED Segment/Common Driver with Controller. Implementacja obejmuje inicjalizację wyświetlacza, konfigurację parametrów transmisji SPI oraz bibliotekę funkcji do wyświetlania informacji o stanie rośliny, poziomie światła i innych parametrach środowiskowych. Wyświetlacz służy jako interfejs użytkownika, prezentując dane w czytelnej formie graficznej.

\subsection{System audio}
System audio inteligentnej donicy składa się z głośnika (oznaczenie na schemacie SP1) oraz układu sterowania LM4811MM (oznaczenie na schemacie U10). Układ ten umożliwia generowanie sygnałów dźwiękowych informujących użytkownika o stanie rośliny. Implementacja obejmuje inicjalizację układu sterowania, konfigurację parametrów dźwięku oraz bibliotekę funkcji do odtwarzania różnych melodii w zależności od poziomu wilgotności gleby.

\subsection{Sterowanie GPIO}
System wykorzystuje piny GPIO mikrokontrolera do sterowania różnymi elementami wykonawczymi donicy. Funkcjonalność ta obejmuje konfigurację kierunku pinów (wejście/wyjście), ustawianie stanów logicznych oraz obsługę przerwań sprzętowych. GPIO jest wykorzystywane między innymi do sterowania diodami sygnalizacyjnymi, wykrywania poziomu wody oraz obsługi przycisków interfejsu użytkownika.

\subsection{Modulacja szerokości impulsu (PWM)}
Implementacja PWM umożliwia precyzyjne sterowanie elementami wykonawczymi wymagającymi sygnału analogowego, takimi jak pompy wody czy wentylatory. Funkcjonalność obejmuje konfigurację częstotliwości sygnału PWM, ustawianie wypełnienia impulsu (duty cycle) oraz dynamiczną zmianę parametrów w zależności od warunków środowiskowych. PWM jest wykorzystywane głównie do regulacji intensywności doświetlania rośliny za pomocą lamp LED.

\subsection{Timer}
System wykorzystuje timery sprzętowe mikrokontrolera do precyzyjnego odmierzania czasu i wykonywania cyklicznych zadań. W przeciwieństwie do PWM, który służy do generowania sygnałów o zmiennym wypełnieniu, timery są wykorzystywane do wywoływania przerwań w określonych odstępach czasu. Funkcjonalność ta obejmuje konfigurację preskalerów, rejestrów porównania oraz obsługę przerwań. Timery są używane do planowania pomiarów, kontroli cykli nawadniania oraz implementacji funkcji oszczędzania energii.

\subsection{Przetwornik analogowo-cyfrowy (ADC)}
System wykorzystuje przetwornik analogowo-cyfrowy do odczytu analogowego wejścia czujnika określanego jako "czujnik wilgotności". Należy zaznaczyć, że termin "wilgotność" jest w tym przypadku skrótem myślowym, ponieważ w rzeczywistości nie mierzymy bezpośrednio wilgotności gleby, a jej przewodność elektryczną, która jest skorelowana z zawartością wody w glebie. Implementacja obejmuje:

\begin{enumerate}
    \item \textbf{Inicjalizacja i konfiguracja przetwornika ADC} - Proces inicjalizacji obejmuje konfigurację rejestrów przetwornika, ustawienie częstotliwości próbkowania, wybór kanału pomiarowego oraz określenie napięcia referencyjnego.
    
    \item \textbf{Odczyt wartości analogowej} - Funkcjonalność ta odpowiada za uruchomienie konwersji analogowo-cyfrowej, odczyt wyniku konwersji oraz obsługę ewentualnych błędów pomiaru.
    
    \item \textbf{Interpretacja danych pomiarowych} - System przetwarza odczytane wartości na względny poziom wilgotności gleby. Proces ten obejmuje kalibrację czujnika (określenie wartości dla suchej i mokrej gleby), filtrowanie zakłóceń oraz konwersję surowych danych na format zrozumiały dla użytkownika.
\end{enumerate}


\end{document}





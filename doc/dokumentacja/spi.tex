\subsection{Magistrala SPI}
System wykorzystuje magistralę SPI z interfejsem SSP1 do komunikacji z wyświetlaczem OLED. 
Piny SPI:
\begin{itemize}
\item \textbf{P0.7} - SCK1 (Serial Clock for SSP1); generowany przez mastera sygnał zegarowy służący do synchronizacji przesyłu danych.
\item \textbf{P0.8} - MISO1 (Main In, Sub Out); służy do przesyłu danych od slave'a do mastera; SPI używamy tylko dla OLED więc MISO nie jest używany;
\item \textbf{P0.9} - MOSI1 (Main Out, Sub In); służy do przesyłu danych od mastera do slave'a;
\item \textbf{P2.2 - SSEL} (Slave Select); umożliwia wybór slave’a do transmisji, jest aktywowany stanem niskim; sterowany przez GPIO;
\end{itemize}
Rejestry magistrali SPI:
\begin{itemize}
    \item \textbf{S0SPCR} - SPI Control Register; rejestr sterujący pracą interfejsu SPI, pozwala m.in. na ustawienie trybu pracy, włączenie SPI i przerwań.
\begin{table}[H]
\centering
\caption{Opis bitów rejestru S0SPCR.}
\vspace{0.5em}
\renewcommand{\arraystretch}{1.2}
\begin{tabular}{|c|c|p{9.5cm}|}
\hline
\textbf{Bity} & \textbf{Symbol} & \textbf{Opis} \\
\hline
0–1     & —            & Zarezerwowane. \\
\hline
2       &BitEnable  & 0 – transmisja zawsze 8-bitowa, 1 – długość słowa określona przez bity 11:8 (\texttt{BITS}). \\
\hline
3       &CPHA     & Faza zegara: 0 – dane próbkowane na pierwszym zboczu, 1 – na drugim zboczu. \\
\hline
4       & CPOL       & Polaryzacja zegara: 0 – SCK aktywny wysoki, 1 – SCK aktywny niski. \\
\hline
5       & MSTR       & Tryb pracy SPI: 0 – tryb Slave, 1 – tryb Master. \\
\hline
6       & LSBF       & Kolejność bitów: 0 – MSB pierwszy, 1 – LSB pierwszy. \\
\hline
7       & SPIE       & Przerwania SPI: 0 – wyłączone, 1 – włączone (na zdarzenia SPIF lub MODF). \\
\hline
8–11    & BITS       & Liczba bitów przesyłanych w jednym transferze (aktywny tylko gdy BitEnable = 1). Patrz tabela poniżej. \\
\hline
12–31   & —            & Zarezerwowane. \\
\hline
\end{tabular}

\vspace{1em}

\begin{tabular}{|c|c|c|c|c|c|c|c|c|}
\hline
1000 & 1001 & 1010 & 1011 & 1100 & 1101 & 1110 & 1111 & 0000\\
\hline
8    & 9    & 10   & 11   & 12   & 13   & 14   & 15 & 16\\
\hline
\end{tabular}
\end{table}

    \item \textbf{S0SPSR} - SPI Status Register; rejestr stanu, zawiera flagi informujące o gotowości transmisji, zakończeniu przesyłu i błędach.

\begin{table}[H]
\centering
\caption{Opis bitów rejestru S0SPSR.}
\vspace{0.5em}
\renewcommand{\arraystretch}{1.2}
\begin{tabular}{|c|c|p{9.5cm}|}
\hline
\textbf{Bity} & \textbf{Symbol} & \textbf{Opis} \\
\hline
0–2     & —        & Zarezerwowane.\\
\hline
3       & ABRT     & Przerwanie transmisji w trybie Slave: 1 – wystąpiło przerwanie; bit czyszczony przez odczytanie tego rejestru. \\
\hline
4       & MODF     & Błąd trybu: 1 – wystąpił błąd trybu. Bit czyszczony przez odczytanie tego rejestru i zapis do rejestru kontrolnego SPI. \\
\hline
5       & ROVR     & Nadpisanie przy odczycie: 1 – wystąpił błąd przepełnienia przy odbiorze. Bit czyszczony przez odczytanie tego rejestru. \\
\hline
6       & WCOL     & Kolizja zapisu: 1 – wystąpiła kolizja przy zapisie. Bit czyszczony przez odczytanie tego rejestru i dostęp do rejestru danych SPI. \\
\hline
7       & SPIF     & Flaga zakończenia transmisji SPI: 1 – zakończono transmisję. Dla Mastera: ustawiana na końcu ostatniego cyklu; dla Slave’a: przy ostatnim zboczu próbkującym SCK. Bit czyszczony przez odczytanie tego rejestru i dostęp do rejestru danych SPI. \\
\hline
8–31    & —        & Zarezerwowane.\\
\hline
\end{tabular}
\end{table}

    
    \item \textbf{S0SPDR} - SPI Data Register; dwukierunkowy rejestr danych, zapis do niego inicjuje transmisję, odczyt zwraca dane odebrane przez SPI.

\begin{table}[H]
\centering
\caption{Opis bitów rejestru S0SPDR.}
\vspace{0.5em}
\renewcommand{\arraystretch}{1.2}
\begin{tabular}{|c|c|p{9.5cm}|}
\hline
\textbf{Bity} & \textbf{Symbol} & \textbf{Opis} \\
\hline
0-7     & DataLow  & Dwukierunkowy port danych SPI. Służy do odczytu danych odebranych lub zapisu danych do wysłania. \newline Wartość domyślna: \texttt{0x00}. \\
\hline
8–15    & DataHigh & Dodatkowe bity transmisji/odbioru, gdy bit 2 rejestru \texttt{SPCR} (BitEnable) = 1 i bity 11:8 \texttt{SPCR} są różne od \texttt{1000} (czyli liczba bitów > 8). \newline Gdy liczba bitów < 16, bardziej znaczące bity przy odczycie mają wartość zero. \newline Wartość domyślna: \texttt{0x00}. \\
\hline
16–31   & —         & Zarezerwowane. \\
\hline
\end{tabular}
\end{table}

    
    \item \textbf{S0SPCCR} - SPI Clock Counter Register; ustala częstotliwość zegara SPI (SCK) poprzez wartość dzielnika.

\begin{table}[H]
\centering
\caption{Opis bitów rejestru S0SPCCR.}
\vspace{0.5em}
\renewcommand{\arraystretch}{1.2}
\begin{tabular}{|c|c|p{9.5cm}|}
\hline
\textbf{Bity} & \textbf{Symbol} & \textbf{Opis} \\
\hline
0–7     & Counter    & Ustawienie licznika zegara SPI0. \newline Wartość musi być parzysta i większa lub równa 8. Określa częstotliwość zegara SPI:  
\[ \text{SCK}_{\text{freq}} = \frac{PCLK}{\text{Counter}} \]
\newline Wartość domyślna: \texttt{0x00}. \\
\hline
8–31    & —          & Zarezerwowane.\\
\hline
\end{tabular}
\end{table}

    
    \item \textbf{S0SPINT} - SPI Interrupt Flag; flaga informująca o wystąpieniu przerwania SPI; może być programowo wyczyszczona.

    \begin{table}[H]
\centering
\caption{Opis bitów rejestru S0SPINT}
\vspace{0.5em}
\renewcommand{\arraystretch}{1.2}
\begin{tabular}{|c|c|p{9.5cm}|}
\hline
\textbf{Bity} & \textbf{Symbol} & \textbf{Opis} \\
\hline
0     & SPIF       & Flaga przerwania SPI. Ustawiana przez interfejs SPI w celu wygenerowania przerwania. \newline Zerowana przez zapisanie jedynki do tego bitu. \newline \\
\hline
1–7   & —          & Zarezerwowane.\\
\hline
8–31  & —          & Zarezerwowane. \\
\hline
\end{tabular}
\end{table}
\end{itemize}
Rejestry SSP:
\begin{itemize}
  \item \textbf{CR0} – Control Register 0; ustala rozmiar słowa danych, tryb protokołu SPI oraz współczynnik dzielnika zegara.
  \item \textbf{CR1} – Control Register 1; konfiguruje tryb master/slave i inne ustawienia pracy kontrolera SSP.
  \item \textbf{DR} – Data Register; rejestr danych – zapis wypełnia FIFO nadawcze, odczyt opróżnia FIFO odbiorcze.
  \item \textbf{SR} – Status Register; rejestr stanu – zawiera flagi informujące o stanie transmisji i buforów FIFO.
  \item \textbf{CPSR} – Clock Prescale Register; dzielnik preskalera zegara – ustala podstawową częstotliwość SSP.
  \item \textbf{IMSC} – Interrupt Mask Set and Clear Register; umożliwia maskowanie (włączanie/wyłączanie) poszczególnych źródeł przerwań.
  \item \textbf{RIS} – Raw Interrupt Status Register; pokazuje status aktywnych przerwań niezależnie od ustawienia masek.
  \item \textbf{MIS} – Masked Interrupt Status Register; pokazuje status przerwań uwzględniający ustawienia masek.
  \item \textbf{ICR} – Interrupt Clear Register (SSPICR); służy do ręcznego kasowania wybranych flag przerwań.
  \item \textbf{DMACR} – DMA Control Register; umożliwia włączenie transmisji danych za pomocą kontrolera DMA.
\end{itemize}

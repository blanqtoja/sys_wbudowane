\subsection{Wyświetlacz OLED}

Wyświetlacz OLED model UG-9664HSWAG01 (oznaczenie na schemacie OLED1) komunikuje się przez interfejs SPI. Wyświetlacz bazuje na sterowniku SSD1305 132 x 64 Dot Matrix OLED/PLED Segment/Common Driver with Controller. Wyświetlacz służy jako interfejs użytkownika, prezentując dane w czytelnej formie graficznej.

\subsubsection{Konfiguracja}
Piny sterujące wyświetlaczem OLED:
\begin{itemize}
\item \textbf{P0.6} – CS (Chip Select); aktywny w stanie niskim;
\item \textbf{P2.7} – D/C (Data/Command); w stanie wysokim dane są traktowane jako dane, w stanie niskim przesyłane do rejestru komend.
\item \textbf{P2.1} - RES (reset); w stanie niskim wywołuje reset i inicjalizację, w trakcie normalnej pracy należy utrzymywać go w stanie wysokim;
\end{itemize}
\subsubsection {Inicjalizacja:}
\begin{enumerate}
    \item Ustawiamy piny 2.1, 2,7 oraz 0.6 w tryb wyjścia za pomocą rejestrów GPIO. %Odwołanie do GPIO rejestrów FIODIR
    \item Ustawiamy pin 2.1 w tryb niskiego stanu aby upewnić się, że wyświetlacz jest wyłączony.
    \item Wysyłamy instrukcje inicjalizujące do wyświetlacza OLED.
    \item Odczekujemy krótki czas przed włączeniem wyświetlacza.
    \item Ustawiamy pin 2.1 w tryb wysokiego stanu aby włączyć wyświetlacz.
\end{enumerate}
\subsubsection {Przesyłanie danych:}
\label{przes}
\begin{enumerate}
    \item Ustawienie pinu CS w stan niski.
    \item Ustawienie pinu D/C w stan wysoki w trybie danych lub w stan niski w trybie komendy.
    \item Przygotowanie struktury transferu.
    \item Inicjacja transmisji SPI.
    \item Ustawienie pinu CS w stan wysoki.
\end{enumerate}

\subsubsection {Rysowanie pojedynczego piksela}
\label{piksel}
\begin{enumerate}
    \item Na podstawie współrzędnej y określany jest numer strony (ang. page address) — każda strona odpowiada 8 poziomym liniom pikseli w pamięci graficznej SSD1305.
    \item Obliczany jest adres kolumny x, przy czym należy uwzględnić fakt, że kontroler SSD1305 obsługuje rozdzielczość 132x64 piksele, natomiast fizyczny wyświetlacz posiada tylko 96x64 piksele, co oznacza konieczność stosowania przesunięcia poziomego podczas adresowania kolumn.
    \item Na podstawie współrzędnej y wyliczany jest numer bitu (0–7), który odpowiada konkretnej pozycji w bajcie strony, a następnie tworzona jest odpowiednia maska bitowa.
    \item W buforze cieniowania wykonywana jest modyfikacja odpowiedniego bajtu: bit jest ustawiany lub czyszczony w zależności od wartości koloru (czarny/biały).
    \item Ustawiany jest adres w pamięci wyświetlacza (komendy: wybór strony, kolumny niskiej i wysokiej).
    \item Bajt danych z bufora cieniowania (zaktualizowany) jest przesyłany przez interfejs SPI do pamięci graficznej wyświetlacza tak jak opisano w \ref{przes}.
\end{enumerate}

\subsubsection{Wyświetlanie pojedynczego znaku}
\begin{enumerate}
    \item Z tablicy font5x7 pobierany jest odpowiedni wzorzec bitowy znaku (bitmapa o wysokości 8 linii).
    \item Dla każdej z 8 linii (wierszy) znaku:
    \begin{itemize}
        \item Wczytywany jest bajt danych określający wzorzec piksela w danym wierszu.
        \item Iteracja po 6 kolumnach piksela w danej linii:
        \begin{itemize}
            \item Jeśli odpowiedni bit w bajcie danych jest ustawiony — wybierany jest kolor pierwszoplanowy, W przeciwnym razie — kolor tła).
            \item Ustawienie danego piksela zgodnie z kolorem i współrzędnymi \ref{piksel}.
        \end{itemize}
    \end{itemize}
    \item Po przerysowaniu całej linii znakowej kursor pionowy jest przesuwany w dół o jeden piksel, a poziomy resetowany do początkowej pozycji.
\end{enumerate}


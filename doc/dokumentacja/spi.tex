\subsection{Magistrala SPI}
System wykorzystuje magistralę SPI z interfejsem SSP1 do komunikacji z wyświetlaczem OLED. 
\subsubsection{Piny SPI:}
\begin{itemize}
\item \textbf{P0.7} - SCK1 (Serial Clock for SSP1); generowany przez mastera sygnał zegarowy służący do synchronizacji przesyłu danych.
\item \textbf{P0.8} - MISO1 (Main In, Sub Out); służy do przesyłu danych od slave'a do mastera; SPI używamy tylko dla OLED więc MISO nie jest używany;
\item \textbf{P0.9} - MOSI1 (Main Out, Sub In); służy do przesyłu danych od mastera do slave'a;
\item \textbf{P2.2 - SSEL} (Slave Select); umożliwia wybór slave’a do transmisji, jest aktywowany stanem niskim; sterowany przez GPIO;
\end{itemize}
\subsubsection {Rejestry magistrali SPI:}
\begin{itemize}
    \item \textbf{S0SPCR} - SPI Control Register; rejestr sterujący pracą interfejsu SPI, pozwala m.in. na ustawienie trybu pracy, włączenie SPI i przerwań.
\begin{table}[H]
\centering
\caption{Opis bitów rejestru S0SPCR.}
\vspace{0.5em}
\renewcommand{\arraystretch}{1.2}
\begin{tabular}{|c|c|p{9.5cm}|}
\hline
\textbf{Bity} & \textbf{Symbol} & \textbf{Opis} \\
\hline
0–1     & —            & Zarezerwowane. \\
\hline
2       &BitEnable  & 0 – transmisja zawsze 8-bitowa, 1 – długość słowa określona przez bity 11:8 (\texttt{BITS}). \\
\hline
3       &CPHA     & Faza zegara: 0 – dane próbkowane na pierwszym zboczu, 1 – na drugim zboczu. \\
\hline
4       & CPOL       & Polaryzacja zegara: 0 – SCK aktywny wysoki, 1 – SCK aktywny niski. \\
\hline
5       & MSTR       & Tryb pracy SPI: 0 – tryb Slave, 1 – tryb Master. \\
\hline
6       & LSBF       & Kolejność bitów: 0 – MSB pierwszy, 1 – LSB pierwszy. \\
\hline
7       & SPIE       & Przerwania SPI: 0 – wyłączone, 1 – włączone (na zdarzenia SPIF lub MODF). \\
\hline
8–11    & BITS       & Liczba bitów przesyłanych w jednym transferze (aktywny tylko gdy BitEnable = 1). Patrz tabela poniżej. \\
\hline
12–31   & —            & Zarezerwowane. \\
\hline
\end{tabular}

\vspace{1em}

\begin{tabular}{|c|c|c|c|c|c|c|c|c|}
\hline
1000 & 1001 & 1010 & 1011 & 1100 & 1101 & 1110 & 1111 & 0000\\
\hline
8    & 9    & 10   & 11   & 12   & 13   & 14   & 15 & 16\\
\hline
\end{tabular}
\end{table}

    \item \textbf{S0SPSR} - SPI Status Register; rejestr stanu, zawiera flagi informujące o gotowości transmisji, zakończeniu przesyłu i błędach.

\begin{table}[H]
\centering
\caption{Opis bitów rejestru S0SPSR.}
\vspace{0.5em}
\renewcommand{\arraystretch}{1.2}
\begin{tabular}{|c|c|p{9.5cm}|}
\hline
\textbf{Bity} & \textbf{Symbol} & \textbf{Opis} \\
\hline
0–2     & —        & Zarezerwowane.\\
\hline
3       & ABRT     & Przerwanie transmisji w trybie Slave: 1 – wystąpiło przerwanie; bit czyszczony przez odczytanie tego rejestru. \\
\hline
4       & MODF     & Błąd trybu: 1 – wystąpił błąd trybu. Bit czyszczony przez odczytanie tego rejestru i zapis do rejestru kontrolnego SPI. \\
\hline
5       & ROVR     & Nadpisanie przy odczycie: 1 – wystąpił błąd przepełnienia przy odbiorze. Bit czyszczony przez odczytanie tego rejestru. \\
\hline
6       & WCOL     & Kolizja zapisu: 1 – wystąpiła kolizja przy zapisie. Bit czyszczony przez odczytanie tego rejestru i dostęp do rejestru danych SPI. \\
\hline
7       & SPIF     & Flaga zakończenia transmisji SPI: 1 – zakończono transmisję. Dla Mastera: ustawiana na końcu ostatniego cyklu; dla Slave’a: przy ostatnim zboczu próbkującym SCK. Bit czyszczony przez odczytanie tego rejestru i dostęp do rejestru danych SPI. \\
\hline
8–31    & —        & Zarezerwowane.\\
\hline
\end{tabular}
\end{table}

    
    \item \textbf{S0SPDR} - SPI Data Register; dwukierunkowy rejestr danych, zapis do niego inicjuje transmisję, odczyt zwraca dane odebrane przez SPI.

\begin{table}[H]
\centering
\caption{Opis bitów rejestru S0SPDR.}
\vspace{0.5em}
\renewcommand{\arraystretch}{1.2}
\begin{tabular}{|c|c|p{9.5cm}|}
\hline
\textbf{Bity} & \textbf{Symbol} & \textbf{Opis} \\
\hline
0-7     & DataLow  & Dwukierunkowy port danych SPI. Służy do odczytu danych odebranych lub zapisu danych do wysłania. \newline Wartość domyślna: \texttt{0x00}. \\
\hline
8–15    & DataHigh & Dodatkowe bity transmisji/odbioru, gdy bit 2 rejestru \texttt{SPCR} (BitEnable) = 1 i bity 11:8 \texttt{SPCR} są różne od \texttt{1000} (czyli liczba bitów > 8). \newline Gdy liczba bitów < 16, bardziej znaczące bity przy odczycie mają wartość zero. \newline Wartość domyślna: \texttt{0x00}. \\
\hline
16–31   & —         & Zarezerwowane. \\
\hline
\end{tabular}
\end{table}

    
    \item \textbf{S0SPCCR} - SPI Clock Counter Register; ustala częstotliwość zegara SPI (SCK) poprzez wartość dzielnika.

\begin{table}[H]
\centering
\caption{Opis bitów rejestru S0SPCCR.}
\vspace{0.5em}
\renewcommand{\arraystretch}{1.2}
\begin{tabular}{|c|c|p{9.5cm}|}
\hline
\textbf{Bity} & \textbf{Symbol} & \textbf{Opis} \\
\hline
0–7     & Counter    & Ustawienie licznika zegara SPI0. \newline Wartość musi być parzysta i większa lub równa 8. Określa częstotliwość zegara SPI:  
\[ \text{SCK}_{\text{freq}} = \frac{PCLK}{\text{Counter}} \]
\newline Wartość domyślna: \texttt{0x00}. \\
\hline
8–31    & —          & Zarezerwowane.\\
\hline
\end{tabular}
\end{table}

    
    \item \textbf{S0SPINT} - SPI Interrupt Flag; flaga informująca o wystąpieniu przerwania SPI; może być programowo wyczyszczona.

    \begin{table}[H]
\centering
\caption{Opis bitów rejestru S0SPINT}
\vspace{0.5em}
\renewcommand{\arraystretch}{1.2}
\begin{tabular}{|c|c|p{9.5cm}|}
\hline
\textbf{Bity} & \textbf{Symbol} & \textbf{Opis} \\
\hline
0     & SPIF       & Flaga przerwania SPI. Ustawiana przez interfejs SPI w celu wygenerowania przerwania. \newline Zerowana przez zapisanie jedynki do tego bitu. \newline \\
\hline
1–7   & —          & Zarezerwowane.\\
\hline
8–31  & —          & Zarezerwowane. \\
\hline
\end{tabular}
\end{table}
\end{itemize}
\subsubsection {Rejestry SSP:}
\begin{itemize}
  \item \textbf{CR0} – Control Register 0; ustala rozmiar słowa danych, tryb protokołu SPI oraz współczynnik dzielnika zegara.

\begin{table}[H]
\centering
\caption{Opis bitów rejestru CR0.}
\vspace{0.5em}
\renewcommand{\arraystretch}{1.2}
\begin{tabular}{|c|c|p{9.5cm}|}
\hline
\textbf{Bity} & \textbf{Symbol} & \textbf{Opis} \\
\hline
0-3     & DSS         & Wybór długości słowa danych przesyłanych w jednej ramce (ang. \textit{Data Size Select}). Wartości 0000–0010 są zarezerwowane i nieobsługiwane. Patrz tabela poniżej. \\
\hline
4-5     & FRF         & Format ramki: \texttt{00} – SPI, \texttt{01} – TI, \texttt{10} – Microwire, \texttt{11} – zarezerwowane (nie używać). \\
\hline
6       & CPOL        & Polaryzacja zegara: 0 – linia SCK niska między transmisjami, 1 – linia SCK wysoka między transmisjami. \\
\hline
7       & CPHA        & Faza zegara: 0 – dane próbkowane przy pierwszym zboczu, 1 – przy drugim zboczu sygnału zegarowego. \\
\hline
8-15    & SCR         & Ustawienie szybkości zegara (ang. \textit{Serial Clock Rate}) – liczba cykli zegara preskalera na jeden bit - 1. Częstotliwość transmisji: $f = \frac{\text{PCLK}}{\text{CPSDVSR} \times (\text{SCR} + 1)}$. \\
\hline
16-31   & —           & Zarezerwowane.\\
\hline
\end{tabular}
\end{table}

\vspace{1em}

\begin{table}[H]
\centering
\caption{Kodowanie pola \texttt{DSS} – długość słowa danych.}
\vspace{0.5em}
\renewcommand{\arraystretch}{1.4}
\begin{tabular}{|c|c|c|c|c|c|c|c|c|c|c|c|c|c|}
\hline
\textbf{DSS} & 0011 & 0100 & 0101 & 0110 & 0111 & 1000 & 1001 & 1010 & 1011 & 1100 & 1101 & 1110 & 1111 \\
\hline
\textbf{Bity} & 4 & 5 & 6 & 7 & 8 & 9 & 10 & 11 & 12 & 13 & 14 & 15 & 16 \\
\hline
\end{tabular}
\end{table}


  
  \item \textbf{CR1} – Control Register 1; konfiguruje tryb master/slave i inne ustawienia pracy kontrolera SSP.

\begin{table}[H]
\centering
\caption{Opis bitów rejestru CR1.}
\vspace{0.5em}
\renewcommand{\arraystretch}{1.2}
\begin{tabular}{|c|c|p{10.2cm}|}
\hline
\textbf{Bit} & \textbf{Symbol} & \textbf{Opis} \\
\hline
0 & LBM & Loop Back Mode: 0 – tryb normalny, 1 – sprzężenie zwrotne (dane wejściowe pochodzą z wyjścia). \\
\hline
1 & SSE & SSP Enable: 0 – kontroler SSP wyłączony, 1 – kontroler SSP włączony (działa na magistrali SPI). \\
\hline
2 & MS & Master/Slave Mode: 0 – tryb Master (SSP steruje SCLK, MOSI, SSEL), 1 – tryb Slave (SSP steruje tylko MISO). Może być ustawiony tylko gdy \texttt{SSE} = 0. \\
\hline
3 & SOD & Slave Output Disable: 0 – wyjście MISO aktywne w trybie Slave, 1 – wyjście MISO zablokowane. Dotyczy tylko trybu Slave. \\
\hline
4-31 & — & Zarezerwowane.\\
\hline
\end{tabular}
\end{table}

  
  \item \textbf{DR} – Data Register; rejestr danych – zapis wypełnia FIFO nadawcze, odczyt opróżnia FIFO odbiorcze.

\begin{table}[H]
\centering
\caption{Opis bitów rejestru DR.}
\vspace{0.5em}
\renewcommand{\arraystretch}{1.2}
\begin{tabular}{|c|c|p{10cm}|}
\hline
\textbf{Bity} & \textbf{Symbol} & \textbf{Opis} \\
\hline
0-15 & DATA & Dane do wysłania lub odebrane dane. \\
\hline
 & & \textbf{Zapis:} Program może zapisać dane do rejestru, gdy bit \texttt{TNF} (Tx FIFO Not Full) w rejestrze statusu jest ustawiony na 1 (FIFO nadajnika nie jest pełne). Jeśli FIFO było puste i kontroler SSP nie jest zajęty, transmisja zaczyna się natychmiast. W przeciwnym wypadku dane zostaną wysłane po zakończeniu poprzednich. Jeśli długość danych jest mniejsza niż 16 bitów, dane powinny być wyjustowane do prawej strony (right-justified). \\
\hline
 & & \textbf{Odczyt:} Program może odczytać dane, gdy bit \texttt{RNE} (Rx FIFO Not Empty) w rejestrze statusu jest ustawiony na 1 (FIFO odbiornika nie jest puste). Odczyt zwraca najstarsze dane w FIFO, wyjustowane do prawej strony jeśli mniej niż 16 bitów. \\
\hline
16-31 & — & Zarezerwowane. \\
\hline
\end{tabular}
\end{table}

  
  \item \textbf{SR} – Status Register; rejestr stanu – zawiera flagi informujące o stanie transmisji i buforów FIFO.

\begin{table}[H]
\centering
\caption{Opis bitów rejestru SR.}
\vspace{0.5em}
\renewcommand{\arraystretch}{1.2}
\begin{tabular}{|c|c|p{9.5cm}|}
\hline
\textbf{Bit} & \textbf{Symbol} & \textbf{Opis} \\
\hline
0 & TFE & Transmit FIFO Empty — bit ustawiony na 1, gdy FIFO nadajnika jest puste, 0 gdy nie jest puste.  \\
\hline
1 & TNF & Transmit FIFO Not Full — bit ustawiony na 1, gdy FIFO nadajnika nie jest pełne, 0 gdy jest pełne.  \\
\hline
2 & RNE & Receive FIFO Not Empty — bit ustawiony na 1, gdy FIFO odbiornika nie jest puste, 0 gdy jest puste.  \\
\hline
3 & RFF & Receive FIFO Full — bit ustawiony na 1, gdy FIFO odbiornika jest pełne, 0 gdy nie jest pełne.  \\
\hline
4 & BSY & Busy — bit ustawiony na 1, gdy kontroler SSP jest zajęty (wysyła/odbiera ramkę lub FIFO nadajnika nie jest puste), 0 gdy jest bezczynny.  \\
\hline
5-31 & — & Zarezerwowane.  \\
\hline
\end{tabular}
\end{table}

  
  \item \textbf{CPSR} – Clock Prescale Register; dzielnik preskalera zegara – ustala podstawową częstotliwość SSP.

\begin{table}[H]
\centering
\caption{Opis bitów rejestru CPSR.}
\vspace{0.5em}
\renewcommand{\arraystretch}{1.2}
\begin{tabular}{|c|c|p{9.5cm}|}
\hline
\textbf{Bity} & \textbf{Symbol} & \textbf{Opis} \\
\hline
0-7 & CPSDVSR & Parzysta wartość z zakresu 2–254, przez którą dzielony jest sygnał zegara SSP\_PCLK, aby uzyskać zegar preskalera. Bit 0 zawsze odczytywany jest jako 0 (rezerwuje parzystość wartości). \\
\hline
8-31 & — & Zarezerwowane.\\
\hline
\end{tabular}
\end{table}

  
  \item \textbf{IMSC} – Interrupt Mask Set and Clear Register; umożliwia maskowanie (włączanie/wyłączanie) poszczególnych źródeł przerwań.

\begin{table}[H]
\centering
\caption{Opis bitów rejestru IMSC.}
\vspace{0.5em}
\renewcommand{\arraystretch}{1.2}
\begin{tabular}{|c|c|p{9.5cm}|}
\hline
\textbf{Bit} & \textbf{Symbol} & \textbf{Opis} \\
\hline
0 & RORIM & Ustawienie tego bitu powoduje włączenie przerwania w przypadku przepełnienia odbiorczego FIFO (Receive Overrun), czyli gdy FIFO jest pełne, a kolejna ramka została całkowicie odebrana. W takim przypadku poprzednie dane w FIFO są nadpisywane. \\
\hline
1 & RTIM & Włącza przerwanie w przypadku wystąpienia Receive Time-out, czyli gdy FIFO nie jest puste, ale dane nie były odczytane przez pewien czas określony jako okres timeout. \\
\hline
2 & RXIM & Włącza przerwanie, gdy odbiorcze FIFO jest co najmniej w połowie zapełnione. \\
\hline
3 & TXIM & Włącza przerwanie, gdy nadające FIFO jest co najmniej w połowie puste. \\
\hline
4-31 & — & Zarezerwowane.\\
\hline
\end{tabular}
\end{table}

  
  \item \textbf{RIS} – Raw Interrupt Status Register; pokazuje status aktywnych przerwań niezależnie od ustawienia masek.

\begin{table}[H]
\centering
\caption{Opis bitów rejestru RIS.}
\vspace{0.5em}
\renewcommand{\arraystretch}{1.2}
\begin{tabular}{|c|c|p{9.5cm}|}
\hline
\textbf{Bit} & \textbf{Symbol} & \textbf{Opis} \\
\hline
0 & RORRIS & Bit ustawiany na 1, gdy nowa ramka została całkowicie odebrana, podczas gdy Rx FIFO było pełne. Wtedy poprzednia ramka jest nadpisywana przez nową. \\
\hline
1 & RTRIS & Bit ustawiany na 1, gdy Rx FIFO nie jest puste i nie zostało odczytane przez okres timeout (określony na podstawie częstotliwości SSP). \\
\hline
2 & RXRIS & Bit ustawiany na 1, gdy Rx FIFO jest co najmniej w połowie zapełnione. \\
\hline
3 & TXRIS & Bit ustawiany na 1, gdy Tx FIFO jest co najmniej w połowie puste. \\
\hline
4-31 & — & Zarezerwowane. \\
\hline
\end{tabular}
\end{table}

  
  \item \textbf{MIS} – Masked Interrupt Status Register; pokazuje status przerwań uwzględniający ustawienia masek.

\begin{table}[H]
\centering
\caption{Opis bitów rejestru MIS.}
\vspace{0.5em}
\renewcommand{\arraystretch}{1.2}
\begin{tabular}{|c|c|p{9.5cm}|}
\hline
\textbf{Bit} & \textbf{Symbol} & \textbf{Opis} \\
\hline
0 & RORMIS & Bit ustawiany na 1, jeśli nowa ramka została całkowicie odebrana podczas gdy Rx FIFO było pełne, i jeśli odpowiednia przerwanie jest włączone. \\
\hline
1 & RTMIS & Bit ustawiany na 1, jeśli Rx FIFO nie jest puste, nie zostało odczytane przez okres timeout oraz przerwanie jest włączone. \\
\hline
2 & RXMIS & Bit ustawiany na 1, jeśli Rx FIFO jest co najmniej w połowie zapełnione i przerwanie jest włączone. \\
\hline
3 & TXMIS & Bit ustawiany na 1, jeśli Tx FIFO jest co najmniej w połowie puste i przerwanie jest włączone. \\
\hline
4-31 & — & Zarezerwowane.\\
\hline
\end{tabular}
\end{table}

  
  \item \textbf{ICR} – Interrupt Clear Register (SSPICR); służy do ręcznego kasowania wybranych flag przerwań.

\begin{table}[H]
\centering
\caption{Opis bitów rejestru ICR.}
\vspace{0.5em}
\renewcommand{\arraystretch}{1.2}
\begin{tabular}{|c|c|p{10cm}|}
\hline
\textbf{Bit} & \textbf{Symbol} & \textbf{Opis} \\
\hline
0 & RORIC & Zapisanie 1 do tego bitu czyści przerwanie „ramka została odebrana, gdy Rx FIFO było pełne”. \\
\hline
1 & RTIC & Zapisanie 1 do tego bitu czyści przerwanie „Rx FIFO nie jest puste i nie zostało odczytane przez okres timeout”. Okres timeout jest taki sam w trybie master i slave i zależy od szybkości SSP: 32 bity przy PCLK / (CPSDVSR × [SCR+1]). \\
\hline
2-31 & — & Zarezerwowane.\\
\hline
\end{tabular}
\end{table}
  
  \item \textbf{DMACR} – DMA Control Register; umożliwia włączenie transmisji danych za pomocą kontrolera DMA.

\begin{table}[H]
\centering
\caption{Opis bitów rejestru DMACR.}
\vspace{0.5em}
\renewcommand{\arraystretch}{1.2}
\begin{tabular}{|c|c|p{10cm}|}
\hline
\textbf{Bit} & \textbf{Symbol} & \textbf{Opis} \\
\hline
0 & RXDMAE & Włącza DMA dla odbiorczego FIFO, gdy ustawiony na 1. Wyłącza DMA dla odbiorczego FIFO, gdy ustawiony na 0. \\
\hline
1 & TXDMAE & Włącza DMA dla nadawczego FIFO, gdy ustawiony na 1. Wyłącza DMA dla nadawczego FIFO, gdy ustawiony na 0. \\
\hline
2-31 & — & Zarezerwowane. \\
\hline
\end{tabular}
\end{table}

  
\end{itemize}

\subsubsection{Inicjalizacja magistrali}
\begin{enumerate}
    \item Ustawienie pinów
    \begin{enumerate}
        \item P0.7 — SCK
        \item P0.8 — MISO
        \item P0.9 — MOSI
        \item P2.2 — SSEL
    \end{enumerate}
    \item Włączenie zasilania SSP1 poprzez ustawienie bitu 10 w rejestrze PCONP.
    \item Konfiguracja ustawień magistrali (tryb pracy: master, format ramki: SPI, liczba bitów danych: 8, polaryzacja i faza zegara: CPOL - wysoki poziom, CPHA - pierwsza krawędź, prędkość zegara: 1 MHz).
    \item Inicjalizacja rejestrów kontrolnych SSP (CR0 i CR1).
    \item Ustawienie prędkości taktowania SSP.
    \item Włączenie magistrali SSP ustawiając bit SSE w rejestrze CR1.
\end{enumerate}

\subsection{Wykorzystanie}
\begin{enumerate}
    \item Konfiguracja pinów SPI (PINSEL), ustawienie parametrów w strukturze SSP\_CFG\_Type, wywołanie SSP\_Init() i SSP\_Cmd() do uruchomienia magistrali.
    \item Wysyłanie danych; ustawienie linii SSEL nisko (aktywacja slave), wpisanie danych do SSP\_DR, oczekiwanie na zakończenie transmisji (BSY=0), odbiór danych równolegle.
    \item Odbieranie danych; Wysłanie tzw. dummy bajtów, oczekiwanie na odebranie danych (RNE=1), odczyt z SSP\_DR.
    \item Zmiana i odczyt rejestrów slave; wysłanie adresu rejestru, wysłanie lub odebranie danych zgodnie z potrzebą.
    \item Koniec transmisji; ustawienie linii SSEL na wysoki poziom.
\end{enumerate}
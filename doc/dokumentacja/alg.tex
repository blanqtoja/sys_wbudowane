\section{Opis głównej pętli programu}

\subsection{Inicjalizacja}
\begin{itemize}
    \item Konfiguracja portów GPIO
    \item Inicjalizacja interfejsów komunikacyjnych (I2C, SPI)
    \item Inicjalizacja przetwornika ADC
    \item Inicjalizacja peryferiów: wyświetlacz OLED, czujnik światła, moduł PWM, czujnik temperatury
    \item Konfiguracja timera SysTick dla odliczania milisekund
\end{itemize}

\subsection{Główna pętla}
W każdej iteracji pętli program wykonuje:

\begin{enumerate}
    \item \textbf{Odczyt temperatury} - pobiera wartość z czujnika i konwertuje na stopnie Celsjusza
    \item \textbf{Wyświetlanie temperatury} - aktualizuje wyświetlacz OLED z aktualną temperaturą
    \item \textbf{Odczyt poziomu oświetlenia} - pobiera wartość z czujnika światła
    \item \textbf{Pomiar wilgotności gleby} - wykorzystuje przetwornik ADC do odczytu z czujnika wilgotności
    \item \textbf{Analiza wilgotności gleby}:
    \begin{itemize}
        \item Gleba sucha (>2801): wyświetla "sucho", dioda świeci żółto-zielono, odtwarza smutną melodię
        \item Gleba wilgotna (1601-2800): wyświetla "w normie", dioda świeci zielono, odtwarza wesołą melodię
        \item Gleba mokra (<1601): wyświetla "mokro", dioda świeci cyjanowo
    \end{itemize}
    \item \textbf{Analiza oświetlenia}:
    \begin{itemize}
        \item Jasno (>2000 lux): wyświetla "jasno"
        \item Normalnie (350-2000 lux): wyświetla "OK"
        \item Ciemno (<350 lux): wyświetla "ciemno"
    \end{itemize}
    \item \textbf{Opóźnienie} - 200ms między iteracjami
\end{enumerate}

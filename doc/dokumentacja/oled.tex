\subsection{Wyświetlacz OLED}

Wyświetlacz OLED model UG-9664HSWAG01 (oznaczenie na schemacie OLED1) komunikuje się przez interfejs SPI. Wyświetlacz bazuje na sterowniku SSD1305 132 x 64 Dot Matrix OLED/PLED Segment/Common Driver with Controller. Wyświetlacz służy jako interfejs użytkownika, prezentując dane w czytelnej formie graficznej.

\subsubsection{Konfiguracja}
Piny sterujące wyświetlaczem OLED:
\begin{itemize}
\item \textbf{P0.6} – CS (Chip Select); aktywny w stanie niskim;
\item \textbf{P2.7} – D/C (Data/Command); w stanie wysokim dane są traktowane jako dane, w stanie niskim przesyłane do rejestru komend.
\item \textbf{P2.1} - RES (reset); w stanie niskim wywołuje reset i inicjalizację, w trakcie normalnej pracy należy utrzymywać go w stanie wysokim;
\end{itemize}
Inicjalizacja:
\begin{enumerate}
    \item Ustawiamy piny 2.1, 2,7 oraz 0.6 w tryb wyjścia za pomocą rejestrów GPIO. %Odwołanie do GPIO rejestrów FIODIR
    \item Ustawiamy pin 2.1 w tryb niskiego stanu aby upewnić się, że wyświetlacz jest wyłączony.
    \item Wysyłamy instrukcje inicjalizujące do wyświetlacza OLED.
    \item Odczekujemy krótki czas przed włączeniem wyświetlacza.
    \item Ustawiamy pin 2.1 w tryb wysokiego stanu aby włączyć wyświetlacz.
\end{enumerate}
Przesyłanie danych:
\begin{enumerate}
    \item Ustawienie pinu CS w stan niski.
    \item Ustawienie pinu D/C w stan wysoki w trybie danych lub w stan niski w trybie komendy.
    \item Ustawienie pinu CS w stan wysoki.
\end{enumerate}



\subsection{System audio}

\subsubsection{Wstęp}
System audio inteligentnej donicy składa się z głośnika (oznaczenie na schemacie SP1) oraz układu sterowania LM4811MM (oznaczenie na schemacie U10). Układ ten umożliwia generowanie sygnałów dźwiękowych informujących użytkownika o stanie rośliny. Implementacja obejmuje inicjalizację układu sterowania, konfigurację parametrów dźwięku oraz bibliotekę funkcji do odtwarzania różnych melodii w zależności od poziomu wilgotności gleby.


\subsubsection{Podstawy Teoretyczne}
\paragraph{Generacja Dźwięku w Mikrokontrolerze}
Dźwięk powstaje poprzez przełączanie pinu GPIO w odpowiedniej częstotliwości, tworząc przebieg prostokątny (square wave).

gdzie $T$ to czas stanu wysokiego/niskiego w mikrosekundach.

\paragraph{Wzmacniacz LM4811}
\begin{itemize}
    \item Sterowany cyfrowo (3 piny: CLK, UP/DN, SHDN).
    \item Zakres głośności: 64 kroki (sterowane impulsami CLK).
    \item Tryb shutdown (SHDN = 0 wyłącza wzmacniacz).
\end{itemize}

\subsubsection{Implementacja}
\paragraph{Konfiguracja Sprzętowa}
\begin{tabular}{|l|l|l|}
\hline
\textbf{Element} & \textbf{Pin LPC1768/9} & \textbf{Funkcja} \\
\hline
LM4811 SIGNAL& P0.26 & Generowanie sygnału dźwiękowego \\
LM4811 CLK & P0.27 & Zegar zmiany głośności \\
LM4811 UP/DN & P0.28 & Kierunek zmiany głośności \\
LM4811 SHDN & P2.13 & Włączanie/wyłączanie wzmacniacza \\
\hline
\end{tabular}

W programie piny te są konfigurowane jako wyjścia przy pomocy funkcji \texttt{GPIO\_SetDir()}:
\begin{verbatim}
GPIO_SetDir(0, 1<<27, 1); // CLK
GPIO_SetDir(0, 1<<28, 1); // UP/DN
GPIO_SetDir(2, 1<<13, 1); // SHDN
GPIO_SetDir(0, 1<<26, 1); // Square wave 
\end{verbatim}

Następnie są ustawiane w stan niski w celu inicjalizacji układu LM4811:
\begin{verbatim}
GPIO_ClearValue(0, 1<<27);
GPIO_ClearValue(0, 1<<28);
GPIO_ClearValue(2, 1<<13);
\end{verbatim}

\paragraph{Odtwarzanie melodii}
Melodie odtwarzane są w zależności od poziomu wilgotności gleby. Odpowiednia funkcja `playSong()` generuje na pinie P0.26 przebieg prostokątny o odpowiedniej częstotliwości i czasie trwania, co przekłada się na dźwięki słyszalne z głośnika. Dwie melodie są używane: „smutna” i „wesoła”.






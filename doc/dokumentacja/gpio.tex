\subsection{Sterowanie GPIO}

\subsubsection{Wstęp}
System wykorzystuje piny GPIO mikrokontrolera do sterowania różnymi elementami wykonawczymi donicy. Funkcjonalność ta obejmuje konfigurację kierunku pinów (wejście/wyjście), ustawianie stanów logicznych oraz obsługę przerwań sprzętowych. GPIO jest wykorzystywane między innymi do sterowania diodami sygnalizacyjnymi, wykrywania poziomu wody oraz obsługi przycisków interfejsu użytkownika.

\subsubsection{Konfiguracja portów GPIO}

W systemie opartym o mikrokontroler LPC1768/9, obsługa pinów GPIO odbywa się dwuetapowo: najpierw należy przypisać pinowi odpowiednią funkcję, a następnie ustawić jego kierunek oraz zarządzać stanem logicznym.

\paragraph{Funkcja pinu}

Piny mikrokontrolera mogą pełnić różne role – od ogólnych wejść/wyjść (GPIO), przez interfejsy komunikacyjne (SPI, UART), aż po funkcje specjalne. Wybór funkcji dokonywany jest za pomocą rejestrów \texttt{PINSEL0} do \texttt{PINSEL9}, z których każdy kontroluje dwa bity odpowiadające konkretnemu pinowi:

\begin{itemize}
    \item \texttt{00} – GPIO (funkcja domyślna),
    \item \texttt{01}, \texttt{10}, \texttt{11} – funkcje alternatywne (specyficzne dla każdego pinu).
\end{itemize}

Pełna lista funkcji alternatywnych znajduje się w dokumentacji mikrokontrolera (sekcja 8.5.1 dokumentu UM10360).

\paragraph{Rejestry GPIO}

Po przypisaniu pinu funkcji GPIO, konfiguracja odbywa się za pomocą zestawu rejestrów \texttt{FIOx}, gdzie \texttt{x} oznacza numer portu (od 0 do 4). Poniżej opis poszczególnych rejestrów:

\begin{itemize}
    \item \textbf{\texttt{FIOxDIR}} – \textit{Direction Register} (Rejestr kierunku):

    Ten rejestr pozwala ustalić kierunek działania każdego pinu portu. 
    \texttt{1} ustawia dany pin jako wyjście (output), natomiast \texttt{0} oznacza wejście (input). 


    \item \textbf{\texttt{FIOxMASK}} – \textit{Mask Register} (Rejestr maski):

    Maskowanie pozwala ignorować niektóre bity przy operacjach odczytu/zapisu. 
    Bity ustawione na \texttt{1} są pomijane przez funkcje korzystające z \texttt{FIOxSET}, \texttt{FIOxCLR} i \texttt{FIOxPIN}. 
    Ustawienie bitu na \texttt{0} oznacza, że operacje na danym pinie są aktywne.


    \item \textbf{\texttt{FIOxPIN}} – \textit{Pin Value Register} (Rejestr stanu pinów):

    Ten rejestr służy zarówno do odczytu, jak i zapisu. 
    Odczyt zwraca aktualny stan logiczny wszystkich pinów danego portu, natomiast zapis nadpisuje piny wyjściowe wartościami logicznymi (jeśli są skonfigurowane jako output).


    \item \textbf{\texttt{FIOxSET}} – \textit{Set Register} (Rejestr ustawień):

    Rejestr służy do ustawiania wybranych pinów wyjściowych na stan wysoki (\texttt{logic 1}). 
    Każdy bit ustawiony na \texttt{1} wymusza stan wysoki na odpowiadającym mu pinie. 
    Ustawienie bitu na \texttt{0} nie zmienia niczego. Nie działa na pinach wejściowych.


    \item \textbf{\texttt{FIOxCLR}} – \textit{Clear Register} (Rejestr zerowania):

    Rejestr ten jest odwrotnością \texttt{FIOxSET}. Pozwala ustawić konkretne piny wyjściowe na stan niski (\texttt{logic 0}). 
    Podobnie jak wcześniej, ustawienie \texttt{1} na wybranym bicie spowoduje wyczyszczenie (ustawienie stanu niskiego) danego pinu. 
    Nie wpływa na piny wejściowe.

\end{itemize}

\subsubsection{Do czego używamy GPIO?}

\begin{table}[H]
\centering
\caption{Przypisanie pinów mikrokontrolera LPC176x do funkcji w projekcie}
\begin{tabular}{@{}lll@{}}
\toprule
\textbf{Pin} & \textbf{Funkcja}                  & \textbf{Opis} \\ \midrule
P0.23        & ADC0                             & Odczyt wilgotności gleby (czujnik analogowy) \\
P0.26        & GPIO                             & Inna funkcja (niezidentyfikowana) \\
P0.27        & LM4811 CLK                       & Sterowanie zegarem układu audio LM4811 \\
P0.28        & LM4811 UP/DN                     & Regulacja głośności LM4811 \\
P2.0         & PWM1.1 (Red)                     & Sterowanie składową czerwoną diody RGB \\
P2.1         & PWM1.2 (Green)                   & Sterowanie składową zieloną diody RGB \\
P2.2         & PWM1.3 (Blue)                    & Sterowanie składową niebieską diody RGB \\
P2.13        & LM4811 SHUTDOWN                  & Włączenie/wyłączenie układu LM4811 \\
\bottomrule
\end{tabular}
\end{table}

